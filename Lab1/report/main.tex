% This is samplepaper.tex, a sample chapter demonstrating the
% LLNCS macro package for Springer Computer Science proceedings;
% Version 2.20 of 2017/10/04
%
\documentclass[runningheads]{llncs}
%
\usepackage{graphicx}
\usepackage[toc,page]{appendix}
\usepackage{listings}
\usepackage{color}
\lstset{
  basicstyle=\ttfamily,
  columns=fullflexible,
%  frame=single,
  breaklines=true,
}

% Used for displaying a sample figure. If possible, figure files should
% be included in EPS format.
%
% If you use the hyperref package, please uncomment the following line
% to display URLs in blue roman font according to Springer's eBook style:
% \renewcommand\UrlFont{\color{blue}\rmfamily}

\begin{document}
%
\title{Reinforcement learning - Lab 1}
%
%\titlerunning{Abbreviated paper title}
% If the paper title is too long for the running head, you can set
% an abbreviated paper title here
%
\author{Dimitri Diomaiuta - 30598109}
%
% \authorrunning{F. Author et al.}
% First names are abbreviated in the running head.
% If there are more than two authors, 'et al.' is used.
%
\institute{University of Southampton}
%% \institute{Princeton University, Princeton NJ 08544, USA \and
%% Springer Heidelberg, Tiergartenstr. 17, 69121 Heidelberg, Germany
%% \email{lncs@springer.com}\\
%% \url{http://www.springer.com/gp/computer-science/lncs} \and
%% ABC Institute, Rupert-Karls-University Heidelberg, Heidelberg, Germany\\
%% \email{\{abc,lncs\}@uni-heidelberg.de}}
%
\maketitle              % typeset the header of the contribution
%
%% \begin{abstract}
%% Searching is one of the oldest artificial intelligence techniques used for problem solving. In this paper we analyze the results and scalability of both uninformed and informed search algorithms.
%% %\keywords{First keyword  \and Second keyword \and Another keyword.}
%% \end{abstract}
%
%
%

\section{Fibonacci}
Fibonacci stuff
\begin{itemize}
\item Show recursion fibonacci expansion
\item Add pseudo code for recursive fibonacci
\item include plots of recursive vs dynamic programming fibonacci
  solvers
\item describe why dynamic programming is a better approach (computing
  same subsolutions only ones).
\end{itemize}
    


\section{Graph search}
Graph search shit
\begin{itemize}
\item Describe what is the purpose of using dynamic programming to
  calculate shortes path
\item Show plots of histograms distribution of node distances (how
  many steps are taken to reach the other nodes)
\end{itemize}

\section{Mountain car}
Mountain car shit
\begin{itemize}
\item Explain what type of reinforcement learning problem this is
  (continous problem rendered as discrete to exploit the Q-table)
\item Explain what happens when changing the exploration rate
\item Show plots to support the claims
\end{itemize}


\section{Appendix A: source code}\label{appendix}
This appendix section contains the source code of the program:
\begin{itemize}
\item Compile with: \textit{javac *.java}
\item Run with: \textit{java -Xss84m Main}
\end{itemize}

%% \subsection{Main.java}
%% \lstinputlisting[language=Java]{../src/Main.java}

%% \subsection{Puzzle.java}
%% \lstinputlisting[language=Java]{../src/Puzzle.java}

%% \subsection{PuzzleManhattanComparator.java}
%% \lstinputlisting[language=Java]{../src/PuzzleManhattanComparator.java}

%% \subsection{SearchState.java}
%% \lstinputlisting[language=Java]{../src/SearchState.java}

%% \subsection{Tile.java}
%% \lstinputlisting[language=Java]{../src/Tile.java}

\end{document}
