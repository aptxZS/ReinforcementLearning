% This is samplepaper.tex, a sample chapter demonstrating the
% LLNCS macro package for Springer Computer Science proceedings;
% Version 2.20 of 2017/10/04
%
\documentclass[runningheads]{llncs}
%
\usepackage{amsmath}
\usepackage{graphicx}
\usepackage[toc,page]{appendix}
\usepackage{listings}
\usepackage{color}
\lstset{
  basicstyle=\ttfamily,
  columns=fullflexible,
%  frame=single,
  breaklines=true,
}

% Used for displaying a sample figure. If possible, figure files should
% be included in EPS format.
%
% If you use the hyperref package, please uncomment the following line
% to display URLs in blue roman font according to Springer's eBook style:
% \renewcommand\UrlFont{\color{blue}\rmfamily}

\begin{document}
%
\title{Reinforcement learning - Lab 2}
%
%\titlerunning{Abbreviated paper title}
% If the paper title is too long for the running head, you can set
% an abbreviated paper title here
%
\author{Dimitri Diomaiuta - 30598109}
%
% \authorrunning{F. Author et al.}
% First names are abbreviated in the running head.
% If there are more than two authors, 'et al.' is used.
%
\institute{University of Southampton}
%% \institute{Princeton University, Princeton NJ 08544, USA \and
%% Springer Heidelberg, Tiergartenstr. 17, 69121 Heidelberg, Germany
%% \email{lncs@springer.com}\\
%% \url{http://www.springer.com/gp/computer-science/lncs} \and
%% ABC Institute, Rupert-Karls-University Heidelberg, Heidelberg, Germany\\
%% \email{\{abc,lncs\}@uni-heidelberg.de}}
%
\maketitle              % typeset the header of the contribution
%
%% \begin{abstract}
%% Searching is one of the oldest artificial intelligence techniques used for problem solving. In this paper we analyze the results and scalability of both uninformed and informed search algorithms.
%% %\keywords{First keyword  \and Second keyword \and Another keyword.}
%% \end{abstract}
%
%
%
\section{Radial basis function}\label{RBF}
In this paper we analyze how to use radial basis function (RBF) to
solve a linear regression and a reinforcement learning problem. A RBF
is a real-valued function $\phi$ whose value depends on the distance
from a point called origin such that $\phi(x) = \phi(||x||)$. RBF
models are particularly useful to generate basis functions from
feature vectors in order to solve function approximation problems via
linear combination. The output of the linear combination is described
by equation \ref{rbf_linear_comb}.
\begin{equation}\label{rbf_linear_comb}
  f(x) = \sum_{j=1}^{J}w_j\phi(||x-c_j||)
\end{equation}
The radial basis function is usually assumed to be Gaussian and,
hence, to be calculated as shown in equation \ref{rbf_gaussian}
\cite{rlbook}.
\begin{equation}\label{rbf_gaussian}
\phi(||x-c_j||) = \exp\left(-\frac{||s-c_j||^2}{2\sigma^2_j}\right)
\end{equation}
As we can observe from equation \ref{rbf_linear_comb}, $J$ defines the
number of centers and, hence, the number of radial basis functions used
in the linear combination and $w_j$ the weights for each function.

\section{Linear regression with RBF}
In this section we solve a linear regression problem by exploiting the
RBF model described in section \ref{RBF}. We solve the problem in a
standard manner as follows:
\begin{itemize}
\item Load a raw dataset
\item Divide into train and test set
\item Given a $J$ compute the design matrix of features using a RBF model
\item Learn the weight vector by stochastic gradient descent on the train set
\item Compute the root mean squared error of the model tested
  on the test set
\item Compare the model with one deriving the weight vector by using the
  pseudoinverse.
\end{itemize}
We performed the steps for different number of radial basis function
($J$) to analyze which one is the model that generalizes better the
pattern of the input data. The input dataset we used for the task is
the UCI \textit{red wine quality} dataset \cite{wine_data}. We tested
the program for $J = \{1, 2, 3, 4, 5, 6, 10\}$. We can observe the
loss function convergence of SGD, the divergence of the best obtained
model from the target data and the train and test set RMSE
respectively in figures \ref{fig1}, \ref{fig2}, \ref{fig3}.
\begin{figure}[!tbp]
  \centering
  \begin{minipage}[b]{0.45\textwidth}
    \includegraphics[width=\textwidth]{img/loss/Figure_1.png}
  \end{minipage}
  \hfill
  \begin{minipage}[b]{0.45\textwidth}
    \includegraphics[width=\textwidth]{img/loss/Figure_4.png}
  \end{minipage}
 \caption{Loss convergence for different $J$}\label{fig1}
\end{figure}
\begin{figure}
    \centering
\includegraphics[width=0.66\textwidth]{img/train_and_test/Figure_1.png}
\caption{Root mean squared error for train and test set}\label{fig2}
\end{figure}
\begin{figure}
    \centering
\includegraphics[width=0.66\textwidth]{img/target_vs_pred/Figure_4.png}
\caption{Best model divergence from target data}\label{fig3}
\end{figure}

We can see, for different $J$ values, that the error is
minimized and that the models converge (Figure \ref{fig1}). Having
more radial basis functions, hence a higher $J$, makes the model
predictions more accurate and closer to the target. Increasing the
$J$, however, is not always resulting in better and more accurate
models that capture the general pattern of the input data. This
problem is well known as the overfitting problem and results in a
model being influenced not only by the pattern but also by the noise
of the data. An overfitted model performs well on the training data
and poorly on the testing data. Figure \ref{fig2} describes this
scenario. We also deduce that the RBF model with $J = 4$
produces the best results. This is confirmed by figure \ref{fig3}, we
can in fact see that the predictions of this model fall close to the
target ones.

\section{Reinforcement learning with RBF}
In a reinforcement learning setting RBF models can be exploited to
derive a value function approximation. Tabular methods are impractical
when dealing with problems that have big state spaces (e.g go has
$10^{170}$ possible states) or that have continous valued features
(tabular methods need preprocessing such tile coding). The first step
is to derive a feature vector representation of a state via RBF ($x(s)$). The
second one is to introduce a weight vector. The value function
approximation is derived by a linear combination of the weight and
feature vector, as shown in equation \ref{value_approx} \cite{rlbook}.
\begin{equation}\label{value_approx}
\hat{v}(s, w) = w^Tx(s) = \sum_{i=1}^{d}w_ix_i(s)  
\end{equation}
The problem now resembles a supervised learning setting and, thus, we
can solve it by updating the weight vector exploiting gradient
descent. The weight update using SGD is calculated in a similar
manner to linear regression, as equation \ref{w_update} shows
\cite{rlbook}.
\begin{equation}\label{w_update}
w_{t+1} = w_t + \alpha\left[U_t - \hat{v}(S_t, w_t)\right]x(S_t)
\end{equation}

Using this knowledge we solved the mountain car problem by
deriving a value function approximation. The steps we took are the
following:
\begin{itemize}
  \item Obtain the value function used by tabular discretization in
    \textit{Lab one}.
  \item Construct feature vector state representation using RBF
    \item Use tabular discretization value function as an unbiased
      estimate ($U_t$)
    \item Compute the weight vector using the pseudoinverse.
\end{itemize}
We performed the steps for different numbers of radial basis
functions. In this case we could compute the weight vector using the
pseudoinverse since we used as unbiased estimate of the target value
function ($U_t$) the values obtain by tabular discretization
stored in \textit{q\_table}. When deriving the value function
approximation on-line the unbiased estimator can be, for
example, the return $G_t$ in Monte Carlo methods or the target
$R_{t+1} +\gamma \hat{v}(S_{t+1}, w)$ in temporal difference
methods. We can observe the centers of the radial basis functions for
$J = 10$, the value function derived via tabular discretization and
the value functions approximation derived with different $J$
respectively in figures \ref{fig4}, \ref{fig5} and \ref{fig6}.
\begin{figure}[!htb]
    \centering
\includegraphics[width=0.60\textwidth]{img/mountain_car/RBF_centers.png}
\caption{RBF centers used to create feature vector state}\label{fig4}
\end{figure}
\begin{figure}[!htb]
    \centering
\includegraphics[width=0.60\textwidth]{img/mountain_car/VFq_table.png}
\caption{q\_table derived value function}\label{fig5}
\end{figure}
\begin{figure}[!htb]
  \centering
  \begin{minipage}[b]{0.45\textwidth}
    \includegraphics[width=\textwidth]{img/mountain_car/VF2.png}
  \end{minipage}
  \hfill
  \begin{minipage}[b]{0.45\textwidth}
    \includegraphics[width=\textwidth]{img/mountain_car/VF10.png}
  \end{minipage}
    \hfill
  \begin{minipage}[b]{0.45\textwidth}
    \includegraphics[width=\textwidth]{img/mountain_car/VF20.png}
  \end{minipage}
    \hfill
  \begin{minipage}[b]{0.45\textwidth}
    \includegraphics[width=\textwidth]{img/mountain_car/VF40.png}
  \end{minipage}
 \caption{Value functions approximation for different $J$}\label{fig6}
\end{figure}
As we can see from figure \ref{fig6}, the data indicate that
increasing the number of radial basis functions results into a more
precise approximation. The figure with $J = 40$ is the one that is
closer to the figure plotting the value function derived by the table
discretization method, see figure \ref{fig5}. Increasing $J$,
though, results in drastically incrementing the computational steps of the
algorithm making this infeasible for very high $J$ values. In this
case a $J$ value of $10$ is enough to drive the car on top of the
mountain. 


\begin{thebibliography}{8}

\bibitem{rlbook}
Sutton, R.S. and Barto, A.G., 2011. Reinforcement learning: An introduction.

\bibitem{wine_data}
Archive.ics.uci.edu. (2019). UCI Machine Learning Repository: Wine Quality Data Set. [online] Available at: \url{https://archive.ics.uci.edu/ml/datasets/Wine+Quality} [Accessed 5 Mar. 2019].

\end{thebibliography}

\end{document}
