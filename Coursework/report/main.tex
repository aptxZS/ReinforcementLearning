% This is samplepaper.tex, a sample chapter demonstrating the
% LLNCS macro package for Springer Computer Science proceedings;
% Version 2.20 of 2017/10/04
%
\documentclass[runningheads]{llncs}
%
\usepackage{amsmath}
\usepackage{graphicx}
\usepackage[toc,page]{appendix}
\usepackage{listings}
\usepackage{color}
\lstset{
  basicstyle=\ttfamily,
  columns=fullflexible,
%  frame=single,
  breaklines=true,
}

% Used for displaying a sample figure. If possible, figure files should
% be included in EPS format.
%
% If you use the hyperref package, please uncomment the following line
% to display URLs in blue roman font according to Springer's eBook style:
% \renewcommand\UrlFont{\color{blue}\rmfamily}

\begin{document}
%
\title{Reinforcement learning - Coursework}
%
%\titlerunning{Abbreviated paper title}
% If the paper title is too long for the running head, you can set
% an abbreviated paper title here
%
\author{Dimitri Diomaiuta - 30598109}
%
% \authorrunning{F. Author et al.}
% First names are abbreviated in the running head.
% If there are more than two authors, 'et al.' is used.
%
\institute{University of Southampton}
%% \institute{Princeton University, Princeton NJ 08544, USA \and
%% Springer Heidelberg, Tiergartenstr. 17, 69121 Heidelberg, Germany
%% \email{lncs@springer.com}\\
%% \url{http://www.springer.com/gp/computer-science/lncs} \and
%% ABC Institute, Rupert-Karls-University Heidelberg, Heidelberg, Germany\\
%% \email{\{abc,lncs\}@uni-heidelberg.de}}
%
\maketitle              % typeset the header of the contribution
%
%% \begin{abstract}
%% Searching is one of the oldest artificial intelligence techniques used for problem solving. In this paper we analyze the results and scalability of both uninformed and informed search algorithms.
%% %\keywords{First keyword  \and Second keyword \and Another keyword.}
%% \end{abstract}
%
%
%

\section{Introduction}

\section{Algorithm design}

\subsection{Game agnostic algorithms}
\subsubsection{Exp3 algorithm}
\subsubsection{Epsilon greedy algorithm}

\subsection{Heuristic based algorithms}
\subsubsection{Stick and follow algorithm}
\subsubsection{EA2 algorithm}


\section{Evaluation}

\section{Conclusion}

\begin{thebibliography}{8}

\bibitem{rlbook}
Sutton, R.S. and Barto, A.G., 2011. Reinforcement learning: An introduction.

\end{thebibliography}

\section{Appendix A: source code}\label{appendix}
This appendix section contains the source code of the implemented program.

%% \subsection{capture\_the\_flag.py}\label{capture_the_flag.py}
%% \lstinputlisting[language=Python]{../capture_the_flag.py}


\end{document}
